\documentclass[conference]{IEEEtran}

\usepackage{amsfonts,amsmath,amssymb}
\usepackage{graphicx}
\usepackage{xcolor}
\usepackage{natbib}

%notations
\newtheorem{thm}{Theorem}[section]
\newtheorem{cor}[thm]{Corollary}
\newtheorem{defn}[thm]{Definition}
\newtheorem{exm}[thm]{Example}
\newtheorem{lem}[thm]{Lemma}
\newtheorem{prop}[thm]{Proposition}
\newtheorem{rem}[thm]{Remark}

\newcommand{\corref}[1]{Corollary~{\rm \ref{#1}}}
\newcommand{\defref}[1]{Definition~{\rm \ref{#1}}}
\newcommand{\exmref}[1]{Example~{\rm \ref{#1}}}
\newcommand{\lemref}[1]{Lemma~{\rm \ref{#1}}}
\newcommand{\propref}[1]{Proposition~{\rm \ref{#1}}}
\newcommand{\prpropref}[1]{Proposition~{\bf \ref{#1}}}
\newcommand{\remref}[1]{Remark~{\rm \ref{#1}}}
\newcommand{\thmref}[1]{Theorem~{\rm \ref{#1}}}
\newcommand{\ththmref}[1]{Theorem~{\bf \ref{#1}}}

\renewcommand{\thesection}{\arabic{section}}
\renewcommand{\theequation}{\thesection.\arabic{equation}}
\newcommand{\beq}[1]{\begin{equation} \label{#1}}
	\newcommand{\eeq}{\end{equation}}
\newcommand{\bea}{\bed\begin{array}{rl}}
	\newcommand{\eea}{\end{array}\eed}
\newcommand{\bed}{\begin{displaymath}}
	\newcommand{\eed}{\end{displaymath}}
\newcommand{\barray}{\begin{array}{ll}}
	\newcommand{\earray}{\end{array}}

\newcommand{\aad}{&\disp}
\newcommand{\ad}{&\!\!\!\disp}
\newcommand{\al}{\alpha}
\newcommand{\be}{\beta}
\newcommand{\disp}{\displaystyle}
\newcommand{\dl}{\delta}
\newcommand{\E}{{\mathbb E}}
\newcommand{\e}{\varepsilon}
\newcommand{\ee}{\Delta}
\newcommand{\la}{\lambda}
\newcommand{\PP}{{\mathbb P}}
\newcommand{\nd}{\noindent}
\newcommand{\sg}{\sigma}


\newcommand{\NM}{{\rm NM}}
\newcommand{\AM}{{\rm AM}}

\newcommand{\A}{{\cal A}}
\newcommand{\F}{{\cal F}}
\newcommand{\I}{{\cal I}}
\newcommand{\M}{{\cal M}}
\newcommand{\U}{{\cal U}}
\newcommand{\X}{{\cal X}}
\newcommand{\II}{{\cal I}}
\newcommand{\HH}{{\cal H}}

\newcommand{\bdd}{\hspace*{-0.08in}{\bf.}\hspace*{0.05in}}

\newcommand{\lbar}{\overline}
\newcommand{\wdt}{\widetilde}
\newcommand{\wdh}{\widehat}
\newcommand{\diag}{{\rm diag}}
\def\th{\theta}
\def\para#1{\vskip 0.4\baselineskip\noindent{\bf #1}}
\def\paraa#1{\vskip 0.4\baselineskip\noindent{\sf #1}}

\def\rr{{\Bbb R}}
\def\({\left(}
\def\){\right)}
\def\one{{\hbox{1{\kern -0.35em}1}}}
\def\op{{\cal L}}

\def\gg{{\cal G}}
\def\sth{\sum^2_{\th=1}}


\begin{document}

\title{CSE 5830 Course Project Proposal}

\author{\IEEEauthorblockN{Xiaohang Ma}
\IEEEauthorblockA{\textit{Department of Mathematics} \\
\textit{University of Connecticut}\\
Storrs, CT, USA \\
xiaohang.ma@uconn.edu}
\and
\IEEEauthorblockN{Shiying Xiao}
\IEEEauthorblockA{\textit{Department of Statistics} \\
\textit{University of Connecticut}\\
Storrs, CT, USA \\
shiying.xiao@uconn.edu}
\and
\IEEEauthorblockN{Xiaohui Yin}
\IEEEauthorblockA{\textit{Department of Statistics} \\
\textit{University of Connecticut}\\
Storrs, CT, USA \\
xiaohui.yin@uconn.edu}
}

\maketitle

\begin{abstract}
In this project, we will integrate powerful deep neural networks and
statistical methods with a probabilistic graphical model (PGM) to effectively
tackle sequence labeling tasks on a heterogeneous medical care dataset.
\end{abstract}


\begin{IEEEkeywords}
probabilistic graphical model
\end{IEEEkeywords}


\section{Introduction}

\citep{huang2015bidirectional}


\bibliographystyle{plain}
\bibliography{refs}

\end{document}
