\documentclass[leqno,aspectratio=169]{beamer}
\usetheme{Madrid}
\usecolortheme{seahorse}

\usepackage[linesnumbered,algoruled,boxed,lined]{algorithm2e}
\usepackage{amsfonts,amsmath,amsthm}
\usepackage{bm}
\usepackage{booktabs}
\usepackage{caption}
\usepackage{color}
\usepackage{graphicx}
\graphicspath{{./}{../image/}} % graphic path
\usepackage{hyperref}
\hypersetup{colorlinks,citecolor=blue,linkcolor=[RGB]{50,50,172}}
\usepackage{multicol}
\usepackage{multirow}
\usepackage[authoryear]{natbib}
\usepackage{setspace}
\usepackage{soul}
\usepackage{textpos}
\usepackage{tikz}

%% notations
\newcommand{\blue}[1]{\textcolor{blue}{#1}}
\newcommand{\red}[1]{\textcolor{red}{#1}}
\newcommand{\bmI}{\bm{I}}
\newcommand{\bmOmega}{\bm{\Omega}}
\newcommand{\bmSigma}{\bm{\Sigma}}
\newcommand{\bmS}{\bm{S}}
\newcommand{\bmW}{\bm{W}}
\newcommand{\bmX}{\bm{X}}
\newcommand{\bmY}{\bm{Y}}
\newcommand{\bmZ}{\bm{Z}}
\newcommand{\bmbeta}{\bm{\beta}}
\newcommand{\hatbmOmega}{\hat{\bm{\Omega}}}
\newcommand{\tr}{\text{tr}}
\DeclareMathOperator*{\argmin}{arg \, min}

% consistent with R manual
\newcommand\pkg[1]{\texttt{#1}}
\let\proglang=\textsf \let\code=\texttt

\setbeamercovered{transparent}
\setbeamertemplate{caption}[numbered]
\setbeamertemplate{enumerate item}[default]
\setbeamertemplate{itemize item}[circle]
\setbeamertemplate{section in toc}[default]
\setbeamertemplate{subsection in toc}[default]

\AtBeginSection[]{
\begin{frame}<beamer>{Overview}
\tableofcontents[currentsection]
\end{frame}}


\title[\textcolor{black}{FCABN}]{
{\bf \normalsize Title}}
%\subtitle[]{}
\author[Xiaohang Ma, Shiying Xiao, Xiaohui Yin]{\bf
	Xiaohang Ma\textsuperscript{1},
	Shiying Xiao\textsuperscript{2},
	Xiaohui Yin\textsuperscript{2}}
\institute[UConn]{
\textsuperscript{1}{\small Department of Mathematics, University of Connecticut}
\\
\textsuperscript{2}{\small Department of Statistics, University of Connecticut}
}
\date[October 7, 2024]{
{\small Storrs, CT} \\
{\small October 7, 2024}}

\begin{document}

\begin{frame}[plain]
\titlepage
\end{frame}

\addtobeamertemplate{frametitle}{}{
\begin{textblock*}{70mm}(.9\textwidth,-0.65cm)
\includegraphics[width=.2\textwidth]{uconn-wordmark-single-blue.png}
\end{textblock*}}


\begin{frame}
\frametitle{Overview}
\tableofcontents
\end{frame}


\section[Introduction]{Introduction}

\begin{frame}
\frametitle{Motivations}
\begin{itemize}
\item Functional magnetic resonance imaging (fMRI), especially resting-state 
fMRI, is crucial for understanding brain interactions and neurological diseases 
like Alzheimer's disease (AD).
\bigskip
\item Recent studies combine fMRI and tau-PET imaging to link brain 
architecture with tau protein accumulation, a key feature of AD.
\bigskip
\item Network-based functional connectivity analysis methods have emerged as a 
powerful tool for exploring interactions among brain regions.
\bigskip
\item A comprehensive examination of the methods has been lacking.
\end{itemize}
\end{frame}


\begin{frame}[allowframebreaks]
\frametitle{References}
\bibliographystyle{chicago}
\bibliography{../manuscript/refs}
\end{frame}

\end{document}