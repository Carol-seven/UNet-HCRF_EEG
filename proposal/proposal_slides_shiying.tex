\documentclass[leqno]{beamer}
\usetheme{Madrid}
\usecolortheme{seahorse}

\usepackage[linesnumbered,algoruled,boxed,lined]{algorithm2e}
\usepackage{amsfonts,amsmath,amsthm}
\usepackage{bm}
\usepackage{booktabs}
\usepackage{caption}
\usepackage{color}
\usepackage{graphicx}
\graphicspath{{./}{../image/}} % graphic path
\usepackage{hyperref}
\hypersetup{colorlinks,citecolor=blue,linkcolor=[RGB]{50,50,172}}
\usepackage{multicol}
\usepackage{multirow}
\usepackage[authoryear]{natbib}
\usepackage{setspace}
\usepackage{soul}
\usepackage{textpos}
\usepackage{tikz}

%% notations
\newcommand{\blue}[1]{\textcolor{blue}{#1}}
\newcommand{\red}[1]{\textcolor{red}{#1}}
\newcommand{\bmI}{\bm{I}}
\newcommand{\bmOmega}{\bm{\Omega}}
\newcommand{\bmSigma}{\bm{\Sigma}}
\newcommand{\bmS}{\bm{S}}
\newcommand{\bmW}{\bm{W}}
\newcommand{\bmX}{\bm{X}}
\newcommand{\bmY}{\bm{Y}}
\newcommand{\bmZ}{\bm{Z}}
\newcommand{\bmbeta}{\bm{\beta}}
\newcommand{\hatbmOmega}{\hat{\bm{\Omega}}}
\newcommand{\tr}{\text{tr}}
\DeclareMathOperator*{\argmin}{arg \, min}

% consistent with R manual
\newcommand\pkg[1]{\texttt{#1}}
\let\proglang=\textsf \let\code=\texttt

\setbeamercovered{transparent}
\setbeamertemplate{caption}[numbered]
\setbeamertemplate{enumerate item}[default]
\setbeamertemplate{itemize item}[circle]
\setbeamertemplate{section in toc}[default]
\setbeamertemplate{subsection in toc}[default]

%\AtBeginSection[]{
%\begin{frame}<beamer>{Overview}
%\tableofcontents[currentsection]
%\end{frame}}


\title[\textcolor{black}{HBAC}]{
{\bf \normalsize Harmful Brain Activity Classification}}
%\subtitle[]{}
\author[Shiying Xiao]{Shiying Xiao}
\institute[UConn]{Department of Statistics, University of Connecticut}
\date[October 7, 2024]{
{\small Storrs, CT} \\
{\small October 7, 2024}}

\begin{document}

\begin{frame}[plain]
\titlepage
\end{frame}

\addtobeamertemplate{frametitle}{}{
\begin{textblock*}{70mm}(.9\textwidth,-0.65cm)
\includegraphics[width=.2\textwidth]{uconn-wordmark-single-blue.png}
\end{textblock*}}


%\begin{frame}
%\frametitle{Overview}
%\tableofcontents
%\end{frame}


%\section[Introduction]{Introduction}

%\begin{frame}
%\frametitle{Data Description}
%\begin{itemize}
%\item This dataset contains electroencephalography (EEG) signals from hospital
%patients, collected by Harvard Medical School. It was originally part of a
%Kaggle competition that ended on April 8, 2024.
%\bigskip
%\item The goal of analyzing this dataset is to detect six patterns of harmful
%brain activity, including seizure (SZ), generalized periodic discharges (GPD),
%lateralized periodic discharges (LPD), lateralized rhythmic delta activity
%(LRDA), and generalized rhythmic delta activity (GRDA).
%\end{itemize}
%\end{frame}


\begin{frame}
\frametitle{Overview of the Problem}
\begin{itemize}
\item Electroencephalography (EEGs) detect brain signals but are time-consuming
to interpret manually.
\bigskip
\item This competition aims to automate detection of harmful brain activity.
\bigskip
\item Targeted areas: neurocritical care, epilepsy treatment, and
drug development.
%\item Detect and classify harmful brain activities from EEG signals.
%\bigskip
%\item Potential to improve neurocritical care.
%\bigskip
%\item Key target: Seizures and other harmful brain activities.
\end{itemize}
\end{frame}


\begin{frame}
\frametitle{Types of Harmful Brain Activities}
\begin{itemize}
\item Seizures (SZ)
\bigskip
\item Generalized Periodic Discharges (GPD)
\bigskip
\item Lateralized Periodic Discharges (LPD)
\bigskip
\item Lateralized Rhythmic Delta Activity (LRDA)
\bigskip
\item Generalized Rhythmic Delta Activity (GRDA)
\bigskip
\item ``Other'' types of harmful brain activity
\end{itemize}
\end{frame}


\begin{frame}
\frametitle{Dataset Overview}
\begin{itemize}
\item EEG Time Series Data: 50-second windows of brain activity.
\bigskip
\item Spectrogram Data: Frequency information of brain activity.
\bigskip
\item Features derived from both datasets help in classification tasks.
\end{itemize}
\end{frame}
%\begin{frame}
%\frametitle{Dataset Overview}
%\begin{itemize}
%\item EEG Time Series Data: 50-second windows of EEG data sampled at 200 Hz,
%with signals from 19 electrodes.
%\bigskip
%\item Spectrogram Data: Frequency content from longer EEG windows, aiding in
%understanding rhythmic brain activity.
%\end{itemize}
%\end{frame}


\begin{frame}
\frametitle{Extracted Features}
\begin{itemize}
\item Relative Band Powers
\bigskip
\item Spectral Edge Frequency
\bigskip
\item Hjorth Parameters (Mobility and Complexity)
\bigskip
\item Statistical Measures: Mean, Standard Deviation, Skewness, Kurtosis.
\end{itemize}
\end{frame}


%\begin{frame}
%\frametitle{Data Preprocessing}
%\begin{itemize}
%\item Filter out noise: Only keep frequencies between 0.5-40 Hz.
%%Filter EEG signals: Remove noise outside 0.5–40 Hz range.
%\bigskip
%\item Focus on the middle 10 seconds of each 50-second window.
%\bigskip
%\item Generate time and frequency-based features for analysis.
%\end{itemize}
%\end{frame}


\begin{frame}
\frametitle{Potential Impact}
\begin{itemize}
\item Automated EEG analysis saves time and reduces errors.
%%Automating EEG analysis will help reduce manual workload.
\bigskip
\item Quick and accurate detection of brain damage can improve patient outcomes.
%%Quick and accurate detection of brain damage can save lives.
\bigskip
\item Significant impact on epilepsy treatment, neurocritical care,
and drug development.
\end{itemize}
\end{frame}


\begin{frame}
\frametitle{Conclusion}
\begin{itemize}
\item The dataset offers rich information for detecting harmful brain activity.
\bigskip
\item Automating harmful brain activity detection can revolutionize
neurocritical care.
\bigskip
\item Encourages further exploration of EEG and spectrogram data to
improve classification models.
\end{itemize}
\end{frame}


%\begin{frame}[allowframebreaks]
%\frametitle{References}
%\bibliographystyle{plain}
%\bibliography{bib}
%\end{frame}

\end{document}
%%% Local Variables:
%%% mode: latex
%%% TeX-master: t
%%% End:
