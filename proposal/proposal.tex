\documentclass[12pt]{article}

\usepackage[margin = 1in]{geometry}
\usepackage{authblk}
\usepackage[inline]{enumitem}
\usepackage[colorlinks=true,citecolor=blue,urlcolor=blue]{hyperref}
\usepackage{natbib}
\usepackage{setspace}
\onehalfspacing

\newcommand{\sx}[1]{\textcolor{red}{(SX: #1)}}

\begin{document}

\title{Proposal: Transformer-CRF Integration for Sequence Labeling on EEG Data}

\author[1]{Xiaohang Ma}
\author[2]{Shiying Xiao}
\author[2]{Xiaohui Yin}

\affil[1]{Department of Mathematics, University of Connecticut}
\affil[2]{Department of Statistics, University of Connecticut}

\date{\today}

\maketitle


\section{Project Overview}


Sequence labeling tasks are pivotal in diverse domains, especially healthcare,
where precise and timely data classification can significantly impact patient
outcomes. Medical datasets, such as physiological signals, often exhibit
complexity and heterogeneity, posing challenges for traditional modeling
techniques. Therefore, developing models capable of extracting meaningful
patterns from these datasets is crucial for accurate diagnostic and prognostic
decisions.


In recent years, deep learning has emerged as a powerful tool for sequence
labeling. Neural network architectures like long short-term memory (LSTM) and
Transformer models have revolutionized sequential data processing.
Additionally, conditional random fields (CRFs) have been employed for modeling
label dependencies, demonstrating promising results.
However, while these methods have shown individual strengths, a comprehensive
approach that leverages their complementary capabilities is essential for
tackling complex, multi-modal datasets.


This project introduces a novel framework that combines deep neural networks
and probabilistic graphical models (PGMs) to enhance sequence labeling
performance. By employing Transformer~\citep{vaswani2017attention} for feature
extraction and integrating them with CRF models,
we aim to improve sequence tagging accuracy. Furthermore, mean-field
approximation techniques are utilized for efficient inference, ensuring a
computationally efficient yet effective approach.
This integration of modern neural network architectures with CRFs offers
a robust and promising solution for complex sequence labeling tasks.


\section{Literature Review}


Deep neural networks, particularly recurrent neural networks (RNNs) like LSTM,
are widely used for sequence labeling due to their ability to capture
long-range dependencies in sequences. Bidirectional LSTM (Bi-LSTM) further
enhances this by processing sequences in both directions, improving context
understanding.
CRFs, on the other hand, are commonly used for structured predictions,
modeling dependencies between neighboring labels to ensure valid output
sequences.
\citet{huang2015bidirectional} pioneered the use of Bi-LSTM integrated with
CRFs for sequence labeling tasks, demonstrating their effectiveness in
extracting features and ensuring label consistency.


However, Bi-LSTMs can still struggle to fully account for structural
dependencies between predicted labels, leading to potential inconsistencies in
the output sequence. Additionally, they typically require large amounts of
labeled data and are computationally expensive, making them challenging to
deploy in resource-constrained environments, such as hospitals.


Transformer-based models, such as BERT~\citep{devlin2019bert},
have emerged as powerful alternatives to traditional RNN-based approaches.
Trained on massive datasets in an unsupervised manner, Transformer can be
fine-tuned for various natural language processing tasks,
including sequence labeling.
\citet{devlin2019bert} directly compared BERT to traditional models, including
Bi-LSTM, and demonstrated its superior performance on tasks like named entity
recognition, highlighting its ability to capture more feature information
than LSTM models.


While CRFs are effective for modeling label dependencies, they often lack
efficient closed-form solutions, especially for complex models beyond
linear-chain CRFs. To address this, \citet{krahenbuhl2011efficient} introduced
the use of mean-field approximation to solve more general CRFs in the context
of complex prediction tasks. By minimizing the Kullback-Leibler (KL)
divergence, mean-field approximation can be reduced to a fixed-point iteration,
providing a faster and more efficient approach for training and inference
while maintaining suitable accuracy. \citet{zheng2015conditional} further
extended this technique by reformulating mean-field approximation for fully
connected CRFs, integrating an RNN structure to capture temporal dependencies.


\section{Data}


The harmful brain activity classification dataset~\citep{jing2023development}
offers a valuable resource for studying brain health. This dataset,
originally used in a Kaggle competition that concluded on April 8th, 2024
(available at
\url{https://www.kaggle.com/competitions/hms-harmful-brain-activity-classification}),
contains electroencephalography (EEG) signals from hospital patients, collected
by Harvard Medical School. The data comes in two primary formats:
\begin{enumerate*}[label = (\roman*)]
\item EEG time series data, which records 50-second windows of brain activity
measured by 19 electrodes placed on the scalp;
\item Spectrogram data, which provides frequency-based representations of the
EEG signals.
\end{enumerate*}


\section{Activities Plan}


\begin{enumerate}
\item \textbf{Probability graphical model}
(Shiying Xiao and Xiaohang Ma)
\begin{itemize}
\item Build the CRF layer for label predition
\item Mean field approximation inference and training of CRF model
\end{itemize}
\item \textbf{Deep learning methods} (Shiying Xiao, Xiaohui Yin and Xiaohang Ma)
\begin{itemize}
\item CNN and RNN layers for feature extraction
\item RNN-CRF layers for label discrimination
\end{itemize}
\item \textbf{Manuscript writing} (Shiying Xiao, Xiaohui Yin and Xiaohang Ma)
\item \textbf{Project presentation} (Shiying Xiao, Xiaohui Yin and Xiaohang Ma)
\end{enumerate}


\bibliographystyle{chicago}
\bibliography{../manuscript/refs}


\end{document}
