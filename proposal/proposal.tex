\documentclass[12pt]{article}

\usepackage[margin = 1in]{geometry}
\usepackage{authblk}
\usepackage[colorlinks=true,citecolor=blue,urlcolor=blue]{hyperref}
\usepackage{natbib}
\usepackage{setspace}
\onehalfspacing

\newcommand{\sx}[1]{\textcolor{red}{(SX: #1)}}

\begin{document}

\title{Proposal:
BERT-CRF Integration for Sequence Labeling on EEG Data \sx{TBD}}

\author[1]{Xiaohang Ma}
\author[2]{Shiying Xiao}
\author[2]{Xiaohui Yin}

\affil[1]{Department of Mathematics, University of Connecticut}
\affil[2]{Department of Statistics, University of Connecticut}

\date{\today}

\maketitle


\section{Project Overview}


Sequence labeling tasks are pivotal in diverse domains, especially healthcare,
where precise and timely data classification can significantly impact patient
outcomes. Medical datasets, such as physiological signals, often exhibit
complexity and heterogeneity, posing challenges for traditional modeling
techniques. Therefore, developing models capable of extracting meaningful
patterns from these datasets is crucial for accurate diagnostic and prognostic
decisions.


In recent years, deep learning has emerged as a powerful tool for sequence
labeling. Neural network architectures like long short-term memory (LSTM) and
transformer models have revolutionized sequential data processing.
Additionally, conditional random fields (CRFs) have been employed for modeling
label dependencies, demonstrating promising results.
However, while these methods have shown individual strengths, a comprehensive
approach that leverages their complementary capabilities is essential for
tackling complex, multi-modal datasets.


This project introduces a novel framework that combines deep neural networks
and probabilistic graphical models (PGMs) to enhance sequence labeling
performance. By employing bidirectional encoder representations from
transformers (BERT) for feature extraction and integrating them with CRF models,
we aim to improve sequence tagging accuracy. Furthermore, mean-field
approximation techniques are utilized for efficient inference, ensuring a
computationally efficient yet effective approach.
This integration of modern neural network architectures with CRFs offers
a robust and promising solution for complex sequence labeling tasks.



\bibliographystyle{chicago}
\bibliography{../manuscript/refs}


\end{document}
